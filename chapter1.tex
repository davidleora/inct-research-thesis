\section{はじめに}
深層強化学習は,Atari,Go,StarCraft,DOTAといった分野で最近の印象的な人工知能の成果の鍵となっている.深層強化学習がロボティクスや自動化に影響を与えるためには,複雑な物理システムの制御における成功を示す必要がある.また,ロボティクスの多くの潜在的な応用は,人間の近くで相互作用し,不正確に指定された人間の規範を尊重しながら行うことを要求する.カーレースは,これらの課題を正確に提起するドメインであり,複雑で非線形なダイナミクスを持つ車両をリアルタイムで制御し,対戦相手の数インチ以内で運用することを要求する.幸いなことに,このドメインには非常に現実的なシミュレーションが存在し,機械学習アプローチの実験に適している\cite{Gran-Turismo-Sophy}.

本研究では Unity を用いてゲーム環境を開発し,深層強化学習を用いてカーレースAIの学習を行っている.この試みは,交通事故における対応策,特に予測できない運転行動への対応として,自動運転技術の発展を目指している.訓練されたAIエージェントは,様々な交通状況下での適切な反応と安全運転技術を学習する.
